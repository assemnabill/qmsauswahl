%  ----------------------------------------------------------------------------
% Dieses Dokument ist eine Kurzversion von:
%
% Copyright (c) 2016 by Burkhardt Renz. All rights reserved.
% Vorlage Abschlussarbeit THM (minimal)
% $Id: vorlage.tex 3811 2016-08-05 12:03:51Z br $
% 
% Der Dokumentstyle wurde von "srcbook" auf "article" geändert, 
% um den Einstieg in LaTex zu erleichtern. 
% Die Unterdateien wurden nicht per \input ins Dokument geholt, 
% sondern direkt eingefügt. Chapters wurden in Sections 
% umbenannt. Das Titelblatt wurde verändert.
% Die Beschreibung der labeling-Umgebung wurde entfernt. ----------------------------------------------------------------------------

\documentclass[11pt]{scrartcl}       % KOMA-Skript für Artikel
	
%% Präambel
\usepackage[english, ngerman]{babel} % deutsche typogr. Regeln + Trenntabelle
\usepackage[T1]{fontenc}             % interner TeX-Font-Codierung
\usepackage{lmodern}                 % Font Latin Modern
\usepackage[utf8]{inputenc}          % Font-Codierung der Eingabedatei
\usepackage[babel]{csquotes}         % Anführungszeichen
\usepackage{graphicx}                % Graphiken
\usepackage{booktabs}                % Tabellen schöner
\usepackage{listingsutf8}            % Listings mit Einstellungen
\lstset{basicstyle=\small\ttfamily,
	tabsize=2,
	basewidth={0.5em,0.45em},
	extendedchars=true}
\usepackage{amsmath}	               % Mathematik
\usepackage[pdftex]{hyperref}       
\hypersetup{
	bookmarksopen=true,
	bookmarksopenlevel=3,
	colorlinks,
	citecolor=blue,
	linkcolor=blue,
}
\usepackage{scrhack}								 % unterdrückt Fehlermeldung von listings

%% Nummerierungstiefen
\setcounter{tocdepth}{3}             % 3 Stufen im Inhaltsverzeichnis
\setcounter{secnumdepth}{3} 		     % 3 Stufen in Abschnittnummerierung

% ----------------------------------------------------------------------------
\begin{document}



%% Titelseite
	
\titlehead{
\includegraphics[width=0.9\textwidth]{img/mni-logo}
}
\title{Einige wesentliche Kriterien für die Auswahl von Software für ein QMS}
\author{Assem Hussein}
\date{Winter 2022}
\maketitle
	



%% Zusammenfassung
\newpage
\begin{quote}
	\vspace*{2cm}

	\begin{center}
		\textbf{\Large\sffamily Zusammenfassung}
	\end{center}

	Dieser Text beschreibt sich in einem gewissen Sinne selbst, ...
\end{quote}
\vspace*{2cm}



%% Verzeichnissse
 \tableofcontents
%% \listoffigures
%% \listoftables
%\lstlistoflistings

\newpage
\section{Einleitung}

\newpage
\section{Hauptteil}
\subsection{Definitionen, bekanntes Wissen}
\subsection{wissenschaftlicher Beitrag}

\newpage
\section{Schluss}



\begin{thebibliography}{99}                    % max. 99 Einträge möglich


\bibitem{lkurz15}
	Marco Daniel, Patrick Gundlach, Walter Schmidt, Jörg Knappen, Hubert
	Partl und Irene Hyna
	\emph{\LaTeXe-Kurzbeschreibung},
	\url{http://mirror.unicorncloud.org/CTAN/info/lshort/german/l2kurz.pdf},
	2015.
\bibitem{knuth99}
	Donald E. Knuth
	\emph{The \TeX book},
  Reading, MA: Addison-Wesley,
	1986.
\bibitem{koma16}
	Markus Kohm
	\emph{Die Anleitung \textsf{KOMA-Script}},
	\url{http://www.komascript.de/~mkohm/scrguide.pdf}
	2016.
\bibitem{lamport94}
  Leslie Lamport
  \emph{\LaTeX: a document preparation system},
  2nd edition,
  Reading, MA: Addison-Wesley,
  1994.
\bibitem{partosch15}
	Günter Partosch
	\emph{Anforderungen an wissenschaftliche Abschlussarbeiten und wie sie
	mit \LaTeX\ gelöst werden können},
	\url{https://www.staff.uni-giessen.de/partosch/unterlagen/abschlussarbeit.pdf},
	2015.
	
\bibitem{partosch15}
	Günter Partosch
	\emph{Anforderungen an wissenschaftliche Abschlussarbeiten und wie sie
	mit \LaTeX\ gelöst werden können},
	\url{https://www.staff.uni-giessen.de/partosch/unterlagen/abschlussarbeit.pdf},
	2015.

\end{thebibliography}


\end{document}
% ----------------------------------------------------------------------------
