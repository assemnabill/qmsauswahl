%  ----------------------------------------------------------------------------
% Dieses Dokument ist eine Kurzversion von:
%
% Copyright (c) 2016 by Burkhardt Renz. All rights reserved.
% Vorlage Abschlussarbeit THM (minimal)
% $Id: vorlage.tex 3811 2016-08-05 12:03:51Z br $
% 
% Der Dokumentstyle wurde von "srcbook" auf "article" geändert, 
% um den Einstieg in LaTex zu erleichtern. 
% Die Unterdateien wurden nicht per \input ins Dokument geholt, 
% sondern direkt eingefügt. Chapters wurden in Sections 
% umbenannt. Das Titelblatt wurde verändert.
% Die Beschreibung der labeling-Umgebung wurde entfernt. ----------------------------------------------------------------------------

\documentclass[11pt]{scrartcl}       % KOMA-Skript für Artikel
\RequirePackage{filecontents}
\usepackage{color,soul}
%% Präambel
\usepackage[english, ngerman]{babel} % deutsche typogr. Regeln + Trenntabelle
\usepackage[T1]{fontenc}             % interner TeX-Font-Codierung
\usepackage{lmodern}                 % Font Latin Modern
\usepackage[utf8]{inputenc}          % Font-Codierung der Eingabedatei
\usepackage[babel]{csquotes}         % Anführungszeichen
\usepackage{graphicx}                % Graphiken
\usepackage{booktabs}                % Tabellen schöner
\usepackage{listingsutf8}            % Listings mit Einstellungen
\usepackage[
backend=biber,			% Biber Compiler
style=alphabetic,  		% alphanumeric Labels
sorting=anyt 			% sort by alphabetic label, name, year, title 
]{biblatex}				% BibLaTex
\addbibresource{References.bib}

\lstset{basicstyle=\small\ttfamily,
	tabsize=2,
	basewidth={0.5em,0.45em},
	extendedchars=true}
\usepackage{amsmath}	               % Mathematik
\usepackage[pdftex]{hyperref}       
\hypersetup{
	bookmarksopen=true,
	bookmarksopenlevel=3,
	colorlinks,
	citecolor=blue,
	linkcolor=blue,
}
\usepackage{scrhack}								 % unterdrückt Fehlermeldung von listings

%% Nummerierungstiefen
\setcounter{tocdepth}{3}             % 3 Stufen im Inhaltsverzeichnis
\setcounter{secnumdepth}{3} 		 % 3 Stufen in Abschnittnummerierung

% ----------------------------------------------------------------------------
\begin{document}


%% Titelseite
	
\titlehead{
\includegraphics[width=0.9\textwidth]{img/mni-logo}
}
\title{Einige wesentliche Kriterien für die Auswahl von Software für ein QMS}
\author{Assem Hussein}
\date{Winter 2021/2022}
\maketitle	


\newpage

 \tableofcontents


\newpage

\section{Einleitung}
Die Auswahl eines wirksamen QMS\footnote{QMS wird näher definiert in Paragraph 3.1 } ist eine nicht triviale Herausforderung, die einen besonderen Ansatz erfordert. Immer mehr Organisationen streben täglich danach, ein QMS einzuführen und sich zertifizieren zu lassen. Heute übersteigt die Zahl der zertifizierten QMS 1,5 Millionen \cite{Leontyuk_2019}. Ein solch gravierender Trend muss sich bei der Erfüllung der gesetzlichen Anforderungen als wirksam erwiesen haben und gleichzeitig die Kundenzufriedenheit erhöhen, was wiederum zu höheren Gewinnen führen soll.
\\
\\
Hierbei werden allgemeine Verfahren für die Softwareauswahl vorgestellt und die Hauptkriterien der Softwareauswahl erläutert. Insbesondere werden die wesentlichen Kriterien für die Auswahl eines QMS im Detail betrachtet.


\section{Allgemeines Verfahren für die Softwareauswahl}
\subsection{Gängige Ansätze}

In der Praxis gibt es zahlreiche Verfahren zur Softwareauswahl. Einige gängige Verfahren sind in Abb. 1 dargestellt. Unabhängig von der Art des Verfahrens sollten einige allgemeine Kriterien für eine gute Softwarelösung erfüllt sein.

\begin{figure}[!htb]
	\centering
	\includegraphics[width=.75\textwidth]{img/Softwareauswahlverfahren.png}
	\caption{Softwareauswahlverfahren. \cite{teich2008richtige} S. 104 5.1}
\label{fig:qms_principles}
\end{figure}

\subsection{Allgemeine Kriterien}
Hier werden Kriterien erläutert, die für die meisten (mittelständischen) Unternehmen wichtig sind. (vgl. \cite{gross2017professionelle} 7.3) 
\begin{enumerate}

\item[] \textbf{Internes Personal} \\
Der Einsatz einer Software soll dazu dienen, die Aufgaben des Mitarbeiters zu erleichtern. Daher ist es notwendig, dass die Software mit wenig Personal zu bedienen ist. Wenn die Software auf viel Personal angewiesen ist (außer bei Aktualisierungsvorgängen), hat sie sich bereits als kontraproduktiv erwiesen.

\item[] \textbf{Anpassungsfähigkeit} \\
Ein Unternehmen oder ein Mitarbeiter soll die Möglichkeit haben, die Software an seine Bedürfnisse anzupassen, ohne Programmierkenntnisse zu benötigen. Formulare, Auswertungen, Analysen und Masken sind häufige Anwendungsfälle, bei denen eine Anpassung wünschenswert wäre. 

\item[] \textbf{Einfluss auf die Weiterentwicklung} \\
Die Software sollte dem Anwender helfen, seine Aufgaben leicht zu erledigen. Aus diesem Grund soll die Meinung des Benutzers berücksichtigt werden. Es sollte ihm möglich sein, Einfluss auf die Weiterentwicklung des Softwareprodukts zu nehmen. Dies soll von einer Version zur anderen zur Verbesserung der Benutzbarkeit dienen und so zur Steigerung der Kundenzufriedenheit führen.

\item[] \textbf{Dokumentation} \\
Ohne eine nachvollziehbare und umfassende Dokumentation ist der Kunde unter Umständen nicht in der Lage, mit Sonderfällen umzugehen und kann an der Erfüllung seiner Aufgabe gehindert werden. Deshalb sollte diese auch für Personal ohne Spezialkenntnisse leicht verständlich sein.

\item[] \textbf{Prozessflexibilität} \\
Prozesse werden aus Schritten gebildet, die normalerweise sequentiell ablaufen. Die Software sollte es ermöglichen, Schritte nach Bedarf zu überspringen, wenn es zu einem Anwendungsfall kommt, bei dem ein solches Überspringen notwendig ist. Diese Flexibilität ist erforderlich, damit der Benutzer aktiv eingreifen kann. Eine Möglichkeit, Prozesse ("Workflows") zu definieren, wäre ebenfalls von Vorteil.

\item[] \textbf{Kalkulierbarkeit von Anpassungen} \\
Der Einsatz von verständlicher Technologien bietet mehr Raum für den Anwender seine Anpassungen in Kosten umzuwandeln und mögliche Anpassungen gegenseitig verglecihen zu können.


\end{enumerate}


\section{Was ist ein QMS?}
\subsection{Definition und Anwendungsgebiet}
Ein Qualitätsmanagementsystem (kurz: QMS) ist eine Methode der Unternehmensführung, die als Werkzeug zur Qualitätssicherung dient. Ziel ist ein systematisches Qualitätsmanagement durch klare Zielsetzung und effizientes Steuern von Prozessen und Ressourcen. Das soll sicherstellen, dass die Interessen aller beteiligten Parteien (vor allem Kunden) verwirklicht werden. Auf ähnliche Weise erläutert \citeauthor{mai2020grundlage} in \citeyear{mai2020grundlage}, was ein QMS ist:
\begin{quotation}
„Der Begriff des Qualitätsmanagementsystems umfasste die wiederkehrenden und regelmäßigen Tätigkeiten zum Führen und Steuern eines Unternehmens hinsichtlich der Befriedigung von Bedürfnissen interessierter Parteien.“\footnote{\cite{mai2020grundlage}, S.65, 3.1}
\end{quotation}
In diesem Sinne hilft ein QMS, die Aktivitäten einer Organisation zu koordinieren, um die Anforderungen von Kunden und Behörden zu erfüllen sowie ihre Wirksamkeit und Rentabilität stetig zu verbessern. Dies soll zu einer dauerhaften Verbesserung der Unternehmensleistung führen.
\\

Bei der Einführung muss das QMS gezielt auf das Produkt oder die erbrachte Dienstleistung zugeschnitten sein, d. h. es ist wichtig, dass es den Anforderungen des Betriebs gerecht wird. Um jedoch eine korrekte Umsetzung zu gewährleisten, gibt es einige allgemeine Richtlinien etwa in Form der ISO 9001:2015 \cite{normungsinstitut2009qualitatsmanagementsysteme}, die helfen sollen, die Implementierung eines QMS zu standardisieren. 

 

\subsection{Die Grundsätze des QMS}

Das am weitesten verbreitete Modell ist ein QMS, dessen Anforderungen
und Empfehlungen in der internationalen Norm ISO 9000 beschrieben sind \cite{Leontyuk_2019}. Diese Grundsätze müssen bei der Einführung eines QMS berücksichtigt werden, damit die Anforderungen der DIN ISO 9001 erfüllt werden können. Die sieben Grundsätze des Qualitätsmanagements (vgl. \cite{brugger2016din} 1.2) sind:


\begin{enumerate}
\item \textbf{Kundenorientierung (Customer focus)} 
\\
Im Prinzip ist ein Produkt (oder eine Dienstleistung) ein Angebot an Kunden, die einen bestimmten Wunsch haben, und daher sollte das Produkt diesen Wunsch erfüllen. D.h. der Erfolg des Produkts ist an die Zufriedenheit des Kunden gebunden.

\item \textbf{Führung (Leadership)} \\
Führungskräfte sollten aktiv in die Umsetzung eines Ziels eingreifen und nicht nur in die Planung. Damit soll sichergestellt werden, dass die Planung konsequent durchgeführt wird.

\item \textbf{Einbeziehung von Personen (Engagement)} \\
Die Mitarbeiter sollten die Sinnhaftigkeit der neuen Vorschriften begreifen. Andernfalls werden sie diese Vorschriften nicht umsetzen, wenn sie ihnen sinnlos erscheinen.

\item \textbf{Prozessorientierter Ansatz (Process)} \\
Wertschaffung ist nur möglich, wenn Prozesse kooperativ ablaufen können. Die Prozesse sollten daher durch Steuerung und Überwachung konsequent verbessert werden.

\item \textbf{Verbesserung (Improvement)} \\
Beim Qualitätsmanagement geht es nicht nur um die Verbesserung der Produktion, sondern auch der Organisation in einem Unternehmen.

\item \textbf{Faktengestützte Entscheidungsfindung (Evidence)} \\
Entscheidungen von Führungskräften sollen eine logische Grundlage haben und nicht der Intuition folgen. Daher sollten im Vorfeld genügend Daten gesammelt werden.

\item \textbf{Beziehungsmanagement (Relationship management)} \\
Die Unternehmensleistung hängt von der Beziehung zwischen den interessierten Parteien ab. Daher sollte diese Beziehung gepflegt werden, um eine bessere Zusammenarbeit zu erreichen. Wenn die Kooperation zwischen den einzelnen Parteien zunimmt, sollte sich dadurch auch die Gesamtleistung des Unternehmens erhöhen.
\end{enumerate}

\begin{figure}[!htb]
	\centering
	\includegraphics[width=.7\textwidth]{img/iso9001.png}
	\caption{Modell eines ISO 9001 QMS. \cite{pfeifer2021masing} S. 154 8.2}
\label{fig:qms_principles}
\end{figure}


\subsection{Warum digitale QMS?}
Das traditionelle QMS ist die Grundlage für ein digitales QMS, jedoch durch Redundanz charakterisiert. Digitale QMS arbeitet in die Gegenrichtung, da es darauf ausgerichtet ist, Redundanz zu vermeiden \cite{ibrahim2019digital}. Digitale QMS unterstützen den abteilungsübergreifenden Datenfluss in einer Organisation auf automatisierte Weise und verringern die Notwendigkeit manueller Übertragungen, die anfällig für menschliche Fehler und zeitaufwändig sind. Digitale QMS ersetzen papiergestützte QMS, da sie auf Echtzeit-Messungen und Feedback-Mechanismen beruhen, die eine zeitnahe Reaktion auf Ausfälle und Fehler ermöglichen. \cite{yeung2003empirical}


\section{Wesentliche Kriterien für die Auswahl eines QMS}

Bei der Auswahl eines QMS sollen neben den allgemeinen Kriterien für die Softwareauswahl auch einige QMS-spezifische Kriterien berücksichtigt werden. Die Kriterien werden in 2 Kategorien unterteilt: unternehmensrelevant und systemrelevant. Unternehmensrelevante Kriterien beziehen sich auf das Geschäftsmodell und sind von Unternehmen zu Unternehmen unterschiedlich. Die systemrelevanten Kriterien hingegen gelten für alle Arten von QMS.

\subsection{Unternehmensrelevante Kriterien}

\begin{enumerate}

\item[] \textbf{Industriebranche} \\
Um die richtige Software zu wählen, ist es wichtig zu wissen, welche Norm das QMS implementieren soll. Die oben erwähnte Norm ISO 9001:2015 ist die am weitesten verbreitete und eignet sich für viele Branchen, aber eine andere QM-Norm kann für eine bestimmte Branche eine bessere Wahl sein. Die Norm AS9100D beispielsweise richtet sich an Unternehmen der Luft- und Raumfahrt. Sie prüft, ob die Produkte mit den Richtlinien einer Industrienorm übereinstimmen. Hier sind einige weitere Beispiele:

\begin{itemize}
\item IATF 16949: Der Schwerpunkt liegt hier auf der Verbesserung des Produktions- und Lieferprozesses in der Automobilherstellung.

\item ISO 13485: Dieses QMS gewährleistet Sicherheitsrichtlinien für medizinische Geräten.

\item ISO 22000: Standard-QMS für Lebensmittelunternehmen. Sie legt Produktions- und Vertriebsrichtlinien zur Gewährleistung der Lebensmittelsicherheit fest.

\item ISO 27001: Unternehmen in der IT-Branche verwenden diese Norm, um die Qualität und Sicherheit von Informationsnetzwerken zu gewährleisten.
\end{itemize}

Je nach Norm kann der Funktionsumfang eines QMS variieren. Daher ist es notwendig, den Kontext zu berücksichtigen, in dem man tätig ist, um die am besten geeignete Wahl zu treffen.

\item[] \textbf{Funktionserfüllung} \\
Was soll optimiert werden? Ist eine Software bereits im Einsatz? Was erwartet das Unternehmen von einer neuen Software? Diese Fragen sind wichtig, denn eine Software sollte zu den Anforderungen passen und nicht umgekehrt (vgl. \cite{gross2017professionelle} 7.4). Für den Fall, dass eine Software bereits im Einsatz ist, sollte ermittelt werden, ob ein neues oder erweitertes QMS mit der alten Technologie kompatibel ist und integrierbar ist. Außerdem soll der Funktionsumfang allgemein festgelegt werden.

\item[] \textbf{Kosten} \\
Ein weiteres wichtiges Kriterium für das suchende Unternehmen sind die Kosten. Dabei ist es zu beachten, dass diese Kosten nicht nur den Preis des Kernprogramms repräsentieren sondern auch damit verbundenen Lizenzkosten der Datenbank, Preisbestandteile, Schulungsangebote, Wartungspreise, Supportangebote (Handbücher, Onlinehilfen, E-Mail, Hotline, Service vor Ort etc.) (vgl. \cite{gross2017professionelle} 3.10) Es muss dabei beachtet werden, dass Softwareanbieter einige Kosten beim Angebot verstecken. (vgl. \cite{teich2008richtige} 5.7.3)

\item[] \textbf{Einhaltung der Anforderungen} \\
Die Unternehmen unterliegen nationalen und internationalen Vorschriften. Wenn zum Beispiel ein Unternehmen in Indien sein Produkt in einem EU-Land verkaufen möchte, muss das Produkt den in diesem Land und in der gesamten EU geltenden Qualitätsvorschriften entsprechen. Ein QMS sollte in der Lage sein, die Einhaltung dieser Anforderungen nachweislich zu bestätigen, um das Produkt handelsfähig zu machen. (vgl. \cite{alexandrova2020Information})

\end{enumerate}

\subsection{Systemrelevante Kriterien}


\begin{enumerate}

\item[] \textbf{Datenverwaltung} \\
Ein QMS sammelt im Umfang seiner Funktionalitäten erhebliche Mengen an Daten. Hinzu kommt die Analyse und anschließende Auswertung der Daten, so dass die Geschäftsprozesse auf Basis dieser Auswertung optimiert werden können. Da es sich hierbei um eine Hauptfunktionalität handelt und die Daten in großen Mengen anfallen, stellt dies eine Herausforderung für ein QMS dar. Ein wichtiges Kriterium ist ein sogenanntes "Big-Data-System", das die Kosten des Datenmanagements durch eine schnellere Erfassung und Analyse der Daten reduziert. (vgl. \cite{alexandrova2020Information})

\item[] \textbf{Bedienungsfreundlichkeit} \\ 
Ein QMS sollte benutzerfreundlich sein, denn es enthält viele komplexe Funktionen, die schnell unübersichtlich werden können. Die Benutzeroberfläche spielt bei der Auswahl eine wichtige Rolle. Es ist jedoch ein Fehler, die Software nur nach der Benutzeroberfläche auszuwählen, ohne andere wichtige Kriterien wie zum Beispiel die Geschwindigkeit zu berücksichtigen. (vgl. \cite{teich2008richtige} 5.9)

Die Benutzeroberfläche soll Mehrsprachigkeit unterstützen, da es sehr wahrscheinlich ist, dass nicht alle Mitarbeiter (vor allem in multinationalen Organisationen) dieselbe Sprache sprechen. Außerdem ist es vorteilhaft, wenn die Benutzeroberfläche konfigurierbar ist. Im Allgemeinen lässt sich die Benutzerfreundlichkeit am besten anhand von Demoversionen beurteilen. (vgl. \cite{pfeifer2021masing} 4.2.3)

\item[] \textbf{Prozessüberwachung} \\
Eine weitere wesentliche Funktion des QMS ist die Überwachung und Kontrolle der Prozesse. Dies geschieht in der Regel mit statistischen Werkzeugen oder der Einschätzung von Experten, was fehleranfällig ist und viel Zeit in Anspruch nimmt. Durch den Einsatz künstlicher Intelligenz kann ein QMS diese Aufgabe mit Hilfe von Deep Learning schneller und genauer erledigen. (vgl. \cite{alexandrova2020Information})
\end{enumerate}


\section{Fazit}

Der Softwareauswahlprozess ist eine komplexe Aufgabe, die von vielen Bewertungskriterien abhängt. Besonders herausfordernd wird diese Aufgabe, wenn es sich um umfassende Systeme wie zum Beispiel ein QMS handelt. Es gibt verschiedene Ansätze zur Softwareauswahl, bei denen die Erfüllung bestimmter Kriterien wichtiger ist als der Ansatz selbst.\\

In Kapitel 2 wurde gezeigt, welche allgemeingültigen Kriterien für die Softwareauswahl zu erfüllen sind, um die Auswahl als gut zu bezeichnen. Im Anschluss daran wird in Kapitel 3 der Begriff QMS erläutert und warum es sinnvoll ist, ein digitales QMS einzusetzen.\\

In Kapitel 4 werden die Kriterien ausführlicher diskutiert und auf die QMS-Auswahl zugeschnitten. Dort wurden einige wesentliche Kriterien für die Auswahl eines QMS erläutert.\\



\newpage
\phantomsection
\addcontentsline{toc}{section}{Literatur}
\printbibliography
\end{document}
% ----------------------------------------------------------------------------
